\documentclass{article}
\usepackage{fullpage}
\usepackage{listings}
\author{Kevin R. Thornton}
\title{Binary file formats}

\lstset { %
    language=R,
    tabsize=4,
%    backgroundcolor=\color{red}, % set backgroundcolor
    basicstyle=\footnotesize,% basic font setting
}
\begin{document}
\maketitle

\section*{Output from TFL2013\_ind}
\subsection*{Phenotypes file}
The format is:
\begin{enumerate}
\item n = An integer representing how many individuals phenotype data are recorded for.
\item n pairs of doubles.  Each pair is the (genetic,random) component of phenotype.
\end{enumerate}

To read in \texttt{R}:

\begin{lstlisting}
f=file(``phenotypes.bin'',''rb'')
#seek to position in f if needed

#Read in the number of individuals
n=readBin(f,''integer'',1)
#Read in the phenotypes
m=matrix(readBin(f,''numeric'',2*n),ncol=2,byrow=TRUE)
\end{lstlisting}

\subsection*{Effect sizes file}

The format is:
\begin{enumerate}
\item n = An integer representing how many causative mutations for which there are records
\item n sets of 4 doubles.  Each set is position, effect size, count in population, and age of allele
\end{enumerate}

To read in \texttt{R}:
\begin{lstlisting}
f=file(``phenotypes.bin'',''rb'')
#seek to position in f if needed

#Read in the number of causative mutations
n=readBin(f,''integer'',1)
#read in data
m=matrix(readBin(f,"numeric",4*n),ncol=4,byrow=T)
\end{lstlisting}

\subsection*{The entire population}
The population is written/read in C++ using the \texttt{fwdpp} function \texttt{write\_binary\_pop} and \texttt{read\_binary\_pop}, respectively.  The policies needed to read and write mutations are called \texttt{mwriter} and \texttt{mreader}, respectively, and are found in this project's header file \texttt{mutation\_with\_age.hpp}.

One does \textbf{not} read in these files in \texttt{R}, especially if the population size is large!  The RAM required would be massive, and it would be slow.
\end{document}
